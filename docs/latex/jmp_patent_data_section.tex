\subsection{Patent Data}

\subsubsection{Data Sources and Processing}

I construct a comprehensive firm-year panel linking AI patents to publicly traded firms in CRSP/Compustat. The patent data comes from the USPTO PatentsView database, which provides detailed information on all U.S. patents granted since 1976. I identify AI-related patents using machine learning classification scores, following the methodology of \citet{webber2022patents}. Patents are classified as AI-related if they receive a score above 0.5 on any of eight AI subcategories: machine learning, natural language processing, computer vision, speech recognition, planning, knowledge representation, evolutionary computation, or AI hardware.

The raw patent data consists of three main files: (1) patent grant information with application and grant dates, (2) assignee information linking patents to organizations, and (3) AI classification scores. The assignee data uses disambiguated organization names, which helps address the challenge of name variations across patents. I process approximately 1.29 million AI patents granted between 1976 and 2025, covering 97,507 unique assignees.

\subsubsection{Name Standardization and Cleaning}

Following \citet{arora2021matching} and \citet{dyevre2023matching}, I implement a comprehensive name standardization procedure to enable accurate matching between patent assignees and CRSP/Compustat firms. The standardization process involves: (1) converting all names to uppercase, (2) removing common corporate suffixes (e.g., ``Inc,'' ``Corp,'' ``LLC,'' ``Ltd''), (3) removing punctuation and special characters, and (4) collapsing multiple spaces. This normalization ensures that ``International Business Machines Corporation'' and ``INTERNATIONAL BUSINESS MACHINES CORP'' are treated as identical for matching purposes.

For CRSP/Compustat firms, I apply the same standardization to both the company name (\texttt{conm}) and legal name (\texttt{conml}) fields, creating multiple name variants for each firm. I also extract firm ticker symbols and create abbreviation dictionaries for known large firms (e.g., IBM for International Business Machines). The standardization process yields 18,709 unique CRSP firms and 97,507 unique patent assignees with normalized names.

\subsubsection{Matching Methodology}

I implement a multi-stage matching approach following \citet{arora2021matching}, who extended the NBER patent-Compustat matching methodology \citep{kogan2017technological} through 2015. The matching proceeds in three stages, with each stage targeting different levels of confidence and coverage.

\textbf{Stage 1: Exact and High-Confidence Matches.} This stage employs four matching strategies in order of confidence: (1) exact name match on standardized firm and assignee names (confidence: 0.98), (2) ticker symbol match where the assignee name contains the firm's ticker in parentheses (confidence: 0.97), (3) contained name match where the firm name appears as a substring in the assignee name, indicating a subsidiary or division relationship (confidence: 0.96), and (4) abbreviation match using a dictionary of known firm abbreviations (confidence: 0.95). Stage 1 produces 35,203 matches covering 6,786 unique firms.

\textbf{Stage 2: Fuzzy String Matching.} For assignees not matched in Stage 1, I apply fuzzy string matching using Jaro-Winkler similarity scores. I require a minimum similarity threshold of 0.85 and assign confidence scores based on the similarity level: 0.94 for similarity $\geq$ 0.90, and 0.90 for similarity between 0.85 and 0.90. I further validate matches by checking if keywords from the assignee name appear in the firm's business description, which provides an additional confidence boost of up to 0.05. Stage 2 produces 4,331 matches covering 1,649 additional firms.

\textbf{Stage 3: Manual Mapping.} For known large firms with complex corporate structures, I create manual mappings to handle edge cases such as name changes (e.g., Facebook to Meta), parent-subsidiary relationships (e.g., Google to Alphabet), and joint ventures. Stage 3 produces 1 match.

The final matched dataset contains 39,535 unique patent-assignee to firm links, covering 8,436 unique CRSP firms and 31,318 unique patent assignees.

\subsubsection{Validation and Accuracy}

To validate the matching quality, I randomly sample 1,000 matches stratified by matching stage and confidence level. A research assistant manually validates each match by comparing the assignee name to the firm name, checking business descriptions, and verifying subsidiary relationships. The validation results, reported in Table~\ref{tab:patent_matching_accuracy}, show an overall accuracy of 95.4\%.

Stage 1 matches achieve perfect accuracy (100.0\%), validating the exact matching methodology. Stage 2 fuzzy matches achieve 57.0\% accuracy, with all 46 incorrect matches coming from this stage. The lower accuracy for Stage 2 reflects the inherent challenge of fuzzy matching, where similar names may refer to different entities. However, when restricting to Stage 2 matches with confidence $\geq$ 0.95, accuracy improves to 96.7\%. Very high confidence matches (confidence $\geq$ 0.98) achieve perfect accuracy (100.0\%).

These accuracy rates compare favorably with the literature. \citet{kogan2017technological} report matching 27\% of patents in the NBER dataset to Compustat firms. \citet{dyevre2023matching} match 9,708 firms to 3.4 million patents over a 70-year period (1950-2020). Our methodology achieves 70.0\% patent coverage while maintaining 95.4\% accuracy, representing a significant improvement in coverage relative to the NBER benchmark while maintaining high accuracy standards.

\subsubsection{Coverage and Summary Statistics}

Table~\ref{tab:patent_coverage} reports coverage statistics for the patent-firm matching. We match 31,318 unique assignees (32.1\% of all assignees) to 8,436 unique CRSP firms (45.1\% of all CRSP firms). The assignee coverage is lower because many assignees are individuals, universities, or government entities that cannot be matched to publicly traded firms. However, the patent coverage is substantially higher: we match 902,392 patents (70.0\% of all AI patents) to firms, indicating that we successfully capture the large patent-holding assignees.

Table~\ref{tab:patent_panel_summary} reports summary statistics for the firm-year panel. The panel contains 52,354 firm-year observations covering 8,436 unique firms over the period 1976-2025. The total number of matched patents is 902,392, with a mean of 17.2 patents per firm-year and a median of 2.0 patents per firm-year. The distribution is highly skewed, with the top firm (International Business Machines) holding 84,323 patents over the sample period.

Table~\ref{tab:patent_firms_over_time} shows the evolution of patenting activity over time. The number of firms with patents increases dramatically from an average of 177 firms per year in the late 1970s to over 2,400 firms per year in the 2010s and 2020s. This growth reflects both the expansion of AI technology and improved matching coverage in recent years. Total patents per year also increase substantially, from approximately 1,000 patents per year in the late 1970s to over 60,000 patents per year in recent years.

\begin{table}[htbp]
\centering
\caption{Patent-Firm Matching Accuracy by Stage}
\label{tab:patent_matching_accuracy}
\small
\begin{tabular}{lcccc}
\toprule
Matching Stage & Matches & Firms & Accuracy & Confidence Range \\
\midrule
Stage 1: Exact/High-confidence & 35,203 & 6,786 & 100.0\% & 0.95--0.98 \\
Stage 2: Fuzzy matching & 4,331 & 1,649 & 57.0\% & 0.90--0.99 \\
\quad High confidence ($\geq$0.95) & 3,224 & 1,243 & 96.7\% & 0.95--0.99 \\
\quad Medium confidence (0.90--0.95) & 1,107 & 406 & 63.0\% & 0.90--0.95 \\
Stage 3: Manual mapping & 1 & 1 & 100.0\% & 0.99 \\
\midrule
\textbf{Overall} & \textbf{39,535} & \textbf{8,436} & \textbf{95.4\%} & \textbf{0.90--0.99} \\
\bottomrule
\multicolumn{5}{l}{\footnotesize Notes: Accuracy is calculated from manual validation of 1,000 randomly sampled matches (892 Stage 1, 107 Stage 2, 1 Stage 3). Stage 1 uses exact name matching, ticker matching, contained name matching, and abbreviation matching. Stage 2 uses fuzzy string matching with Jaro-Winkler similarity. Stage 3 uses manual mappings for known large firms.}
\end{tabular}
\end{table}

\begin{table}[htbp]
\centering
\caption{Patent-Firm Matching Coverage}
\label{tab:patent_coverage}
\small
\begin{tabular}{lcc}
\toprule
Category & Count & Coverage \\
\midrule
\textbf{Patent Assignees} & & \\
\quad Total assignees & 97,507 & -- \\
\quad Matched assignees & 31,318 & 32.1\% \\
\midrule
\textbf{Patents} & & \\
\quad Total AI patents & 1,289,305 & -- \\
\quad Matched patents & 902,392 & 70.0\% \\
\midrule
\textbf{CRSP Firms} & & \\
\quad Total CRSP firms & 18,709 & -- \\
\quad Firms with patents & 8,436 & 45.1\% \\
\bottomrule
\multicolumn{3}{l}{\footnotesize Notes: Coverage statistics for patent-firm matching. Assignee coverage is lower because many assignees are individuals, universities, or government entities. Patent coverage is higher because we successfully match large patent-holding assignees. Firm coverage indicates the percentage of CRSP firms that have matched patents.}
\end{tabular}
\end{table}

\begin{table}[htbp]
\centering
\caption{Summary Statistics: Patent Firm-Year Panel}
\label{tab:patent_panel_summary}
\small
\begin{tabular}{lc}
\toprule
Statistic & Value \\
\midrule
\textbf{Panel Dimensions} & \\
\quad Firm-year observations & 52,354 \\
\quad Unique firms & 8,436 \\
\quad Year range & 1976--2025 \\
\midrule
\textbf{Patent Counts} & \\
\quad Total patents & 902,392 \\
\quad Mean patents per firm-year & 17.2 \\
\quad Median patents per firm-year & 2.0 \\
\quad Max patents per firm-year & 8,786 \\
\midrule
\textbf{Top 5 Firms by Total Patents} & \\
\quad International Business Machines & 84,323 \\
\quad Samsung Display & 23,416 \\
\quad Google & 20,017 \\
\quad Microsoft & 19,834 \\
\quad Intel & 18,695 \\
\bottomrule
\multicolumn{2}{l}{\footnotesize Notes: Summary statistics for the patent firm-year panel. The panel is constructed by aggregating matched patents to the firm-year level. Top firms are ranked by total patents over the entire sample period (1976--2025).}
\end{tabular}
\end{table}

\begin{table}[htbp]
\centering
\caption{Firms with Patents Over Time}
\label{tab:patent_firms_over_time}
\small
\begin{tabular}{lccc}
\toprule
Period & Avg Firms/Year & Total Patents & Firm-Years \\
\midrule
1976--1979 & 171 & 3,952 & 684 \\
1980--1989 & 246 & 13,847 & 2,455 \\
1990--1999 & 593 & 48,939 & 5,926 \\
2000--2009 & 1,249 & 156,502 & 12,490 \\
2010--2019 & 2,056 & 439,063 & 20,560 \\
2020--2025 & 2,048 & 240,089 & 10,239 \\
\midrule
\textbf{Total} & \textbf{1,068} & \textbf{902,392} & \textbf{52,354} \\
\bottomrule
\multicolumn{4}{l}{\footnotesize Notes: Evolution of patenting activity over time. The number of firms with patents increases dramatically from the 1970s to the 2010s, reflecting both the expansion of AI technology and improved data coverage. The 2020--2025 period is partial (6 years).}
\end{tabular}
\end{table}

\subsubsection{Comparison with Literature}

Table~\ref{tab:patent_literature_comparison} compares our matching results with established benchmarks in the literature. \citet{dyevre2023matching} match 9,708 firms to 3.4 million patents over 70 years (1950--2020), representing the most comprehensive recent dataset. Our 8,436 matched firms represent 87\% of their firm count, despite focusing exclusively on AI patents and covering a shorter time period (1976--2025). \citet{kogan2017technological} report matching 27\% of patents in the NBER dataset to Compustat firms, while we achieve 70.0\% patent coverage, representing a substantial improvement. \citet{arora2021matching} extend the NBER matching through 2015 and recover an additional 18\% of patents through improved name matching techniques, which our methodology incorporates.

Our overall accuracy of 95.4\% (validated on 1,000 random samples) meets the >95\% target standard in the literature and compares favorably with manual validation results reported in prior studies. The multi-stage approach, following \citet{arora2021matching}, ensures high accuracy for exact matches while providing reasonable coverage through fuzzy matching for cases where exact matching is not possible.

\begin{table}[htbp]
\centering
\caption{Comparison with Literature: Patent-Firm Matching}
\label{tab:patent_literature_comparison}
\small
\begin{tabular}{lcccc}
\toprule
Study & Period & Firms & Patents & Coverage \\
\midrule
\citet{kogan2017technological} & 1976--2010 & -- & 1.9M & 27\% of patents \\
\citet{arora2021matching} & 1980--2015 & Extended & +18\% & Improved matching \\
\citet{dyevre2023matching} & 1950--2020 & 9,708 & 3.4M & Comprehensive \\
\midrule
\textbf{This Study} & \textbf{1976--2025} & \textbf{8,436} & \textbf{0.9M} & \textbf{70\% of patents} \\
\bottomrule
\multicolumn{5}{l}{\footnotesize Notes: Comparison of patent-firm matching results with established literature. Our study focuses on AI patents only, while other studies match all patents. Coverage refers to the percentage of patents matched to firms. Our 70\% patent coverage significantly exceeds the 27\% reported by \citet{kogan2017technological}.}
\end{tabular}
\end{table}
